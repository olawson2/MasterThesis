\chapter{Experimental Validation}

This section details the apparatus and the experiments used for validating the developed interconnection device, as well as explaining the rationale behind the particular experiments.

\section{Implementation of the Noisy Microgrids}

Two synchronous generators connected to individual loads are utilized to form the microgrids. They are voltage sources, and the elements required to form a microgrid are a voltage source and a load \cite{Blaabjerg_Grid_Synchronization}.

\textcolor{blue}{
\begin{itemize}
    \item Implementation of droop control in the synchronous machine
    \item Voltage and frequency characteristics chosen (and rationale)
    \item Diagrams
    \item Choices behind the particular settings
\end{itemize}
}

\section{Implementation of the Interconnection}

\textcolor{blue}{
\begin{itemize}
    \item Description of the Semikron inverters
    \item Implementation of droop in the inverter modules
    \item Extra steps taken to ensure experiment worked 
    \item Diagrams
\end{itemize}
}

Two Semikron bidirectional converters, repurposed to be utilized as individual rectifiers and/or inverters, are to be utilized to implement the interconnection.

\section{Experiments}

\textcolor{blue}{
\begin{itemize}
    \item Experiment 1: Formation of droop-controlled "microgrids"
    \item Experiment 2: Formation of interconnection 
    \item Experiment 3: Demonstration of Transfer of specified power amount
\end{itemize}
}